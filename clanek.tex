% Zadnja posodobitev: 14. 1. 2022
\documentclass[twoside,11pt]{article}
\usepackage[slovene]{babel}
\usepackage[utf8]{inputenc}
\usepackage{graphicx}
\usepackage[frame]{matrika}
\usepackage{mathtools}
\usepackage{epstopdf}
\usepackage{units}
% Po potrebi se lahk dodajo drugi standardni paketi, ki ne spreminjajo izgleda dokumenta


\begin{document}

\MAT{1}{11}{2024}
\naslov{Brownovo gibanje oz. Wienerjev proces}

\avtor{Anej Rozman}

\institucija{Fakulteta za matematiko in fiziko \\ Univerza v Ljubljani}

\klasifikacija{~} 
\izvlecek{Cilj clanka je predstaviti Brownovo gibanje v kontekstu verjetnosti kjer je znan pod imenom Weinerjev proces. Odlocil sem se da bom v nadaljevanju 
uporabljal izraz Brownovo gibanje, saj je ta izraz bolj znan v slovenskem prostoru. }
\title{Brownian motion or Wiener process}
\abstract{The aim of the article is to present Brownian motion in the context of probability where it is known as the Wiener process. I have decided to use the term Brownian motion since 
this term is more known in the slovenian space.}

\glava\baselineskip=14.5pt

\smallskip

\section{Uvod}

\section{Poglavje}



\begin{thebibliography}{99}

\bibitem{1} E.A. Cornell, C.E. Wieman, M.R. Matthews, J.R. Ensher in M.H. Anderson, \emph{Observation of Bose-Einstein Condensation in a Dilute Atomic Vapor}, Science \textbf{269} (1995), 198--201. 
\bibitem{2} W. Ketterle, D.M. Kurn, D.S. Durfee, N.J. van Druten, M.R. Andrews, M.-O. Mewes in K.B. Davis, \emph{Bose-Einstein Condensation in a Gas of Sodium Atoms}, Physics Review Letters \textbf{75} (1995), 3969--3973. 


\end{thebibliography}

\end{document}
